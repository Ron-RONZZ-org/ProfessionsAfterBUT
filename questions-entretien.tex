\documentclass[a4paper,11pt]{article}
\usepackage[utf8]{inputenc}
\usepackage[T1]{fontenc}
\usepackage[french]{babel}
\usepackage{geometry}
\usepackage{enumitem}
\usepackage{titlesec}
\usepackage{hyperref}

\geometry{margin=2.5cm}

\titleformat{\section}{\Large\bfseries}{\thesection}{1em}{}
\titleformat{\subsection}{\large\bfseries}{\thesubsection}{1em}{}

\hypersetup{
    colorlinks=true,
    linkcolor=blue,
    urlcolor=blue,
    pdftitle={Questions d'entretien - Professions après BUT GIM},
    pdfauthor={BUT Génie Industriel et Maintenance}
}

\begin{document}

\title{\textbf{Guide d'entretien} \\ 
\large Enseignants-Chercheurs \\ 
\normalsize Métiers après BUT Génie Industriel et Maintenance}
\author{}
\date{}

\maketitle

\section*{Introduction}

Ce guide d'entretien a pour objectif de recueillir votre expérience et vos observations concernant les parcours professionnels des diplômés du BUT Génie Industriel et Maintenance. Vos réponses nous aideront à mieux informer nos étudiants sur les métiers qui s'offrent à eux.

\vspace{0.5cm}

\section{Parcours et Expérience Professionnelle}

\begin{enumerate}[leftmargin=*]
    \item Pouvez-vous nous présenter brièvement votre parcours professionnel et votre domaine de recherche actuel ?
    
    \item Quel est votre lien avec le secteur industriel (collaborations, projets de recherche, interactions professionnelles) ?
    
    \item Quels sont les métiers que vous avez pu observer ou avec lesquels vous avez collaboré dans votre carrière ?
\end{enumerate}

\section{Qualités et Compétences Nécessaires}

\begin{enumerate}[leftmargin=*]
    \item Selon votre expérience, quelles sont les principales qualités humaines recherchées dans les métiers liés au génie industriel et à la maintenance ?
    
    \item Quelles compétences techniques sont essentielles pour réussir dans ces domaines ?
    
    \item Quelles compétences transversales (communication, gestion de projet, travail en équipe, etc.) sont particulièrement valorisées ?
    
    \item Comment ces compétences ont-elles évolué au cours des dernières années avec la transformation numérique et l'Industrie 4.0 ?
\end{enumerate}

\section{Expérience Personnelle et Quotidien Professionnel}

\subsection{Journée type}

\begin{enumerate}[leftmargin=*]
    \item Pouvez-vous décrire une journée type dans votre métier d'enseignant-chercheur ?
    
    \item Quelle est la répartition approximative de votre temps entre l'enseignement, la recherche, l'administration et les autres activités ?
    
    \item Quels types d'interactions avez-vous au quotidien (avec des étudiants, des collègues, des industriels, etc.) ?
\end{enumerate}

\subsection{Ressentis et Réflexions}

\begin{enumerate}[leftmargin=*]
    \item Quels sont les points forts de votre métier ? Ce qui vous motive au quotidien ?
    
    \item Quels sont les aspects les plus challengeants ou les points faibles de votre profession ?
    
    \item Qu'est-ce que vous appréciez particulièrement dans votre travail ?
    
    \item Y a-t-il des aspects de votre métier que vous appréciez moins ou qui peuvent être difficiles ?
    
    \item Comment décririez-vous votre charge de travail ? Est-elle équilibrée ou y a-t-il des périodes particulièrement intenses ?
    
    \item Quel est l'équilibre entre vie professionnelle et vie personnelle dans votre métier ?
    
    \item Quelles sont les principales sources de satisfaction dans votre travail ?
    
    \item Quelles sont les principales difficultés ou frustrations que vous rencontrez ?
\end{enumerate}

\section{Rémunération}

\begin{enumerate}[leftmargin=*]
    \item Pouvez-vous nous donner une idée générale de la grille de rémunération pour un enseignant-chercheur en France ?
    
    \item Comment évolue la rémunération avec l'expérience et les responsabilités ?
    
    \item Existe-t-il des compléments de rémunération (primes, heures supplémentaires, projets de recherche) ?
    
    \item Pour les métiers du secteur industriel que vous connaissez, avez-vous une idée des fourchettes salariales selon les niveaux d'expérience ?
    
    \item Comment se comparent les rémunérations entre le secteur académique et le secteur industriel dans votre domaine ?
\end{enumerate}

\section{Évolutions de Carrière Possibles}

\begin{enumerate}[leftmargin=*]
    \item Quelles sont les principales évolutions de carrière possibles pour un enseignant-chercheur ?
    \begin{itemize}
        \item Maître de conférences $\rightarrow$ Professeur d'université
        \item Responsabilités administratives
        \item Autres perspectives
    \end{itemize}
    
    \item Avez-vous des exemples d'enseignants-chercheurs qui ont évolué vers d'autres métiers (médiateur scientifique, ingénieur R\&D, consultant, etc.) ?
    
    \item Quelles évolutions de carrière recommanderiez-vous pour un jeune diplômé du BUT GIM qui souhaiterait poursuivre dans l'enseignement supérieur et la recherche ?
    
    \item Quelles sont les passerelles possibles entre le monde académique et l'industrie ?
    
    \item Quelles formations complémentaires ou qualifications sont nécessaires pour progresser dans la carrière académique ?
\end{enumerate}

\section{Métiers Accessibles après BUT GIM}

\begin{enumerate}[leftmargin=*]
    \item Quels sont les principaux métiers que vous avez pu observer chez vos anciens étudiants diplômés du BUT GIM ?
    
    \item Quels métiers recommanderiez-vous particulièrement à un étudiant passionné par le génie industriel et la maintenance ?
    
    \item Quels sont les secteurs d'activité les plus dynamiques actuellement pour recruter des diplômés en génie industriel et maintenance ?
    
    \item Observez-vous de nouveaux métiers émergeants dans ces domaines ?
    
    \item Quels conseils donneriez-vous à un étudiant pour maximiser ses chances d'insertion professionnelle ?
\end{enumerate}

\section{Conseils et Perspectives}

\begin{enumerate}[leftmargin=*]
    \item Quel message souhaiteriez-vous transmettre aux étudiants actuels du BUT GIM concernant leur avenir professionnel ?
    
    \item Selon vous, quelles sont les tendances qui vont façonner les métiers du génie industriel et de la maintenance dans les prochaines années ?
    
    \item Y a-t-il des aspects importants que nous n'avons pas abordés et que vous souhaiteriez partager ?
\end{enumerate}

\vspace{1cm}

\section*{Conclusion}

Merci beaucoup pour votre participation et le temps que vous avez consacré à cet entretien. Vos réponses nous seront précieuses pour guider nos étudiants dans leurs choix de carrière.

\vspace{0.5cm}

\noindent\rule{\textwidth}{0.4pt}

\vspace{0.3cm}

\noindent\textit{Date de l'entretien : \underline{\hspace{4cm}}}

\noindent\textit{Nom de l'enseignant-chercheur : \underline{\hspace{6cm}}}

\noindent\textit{Institution : \underline{\hspace{8cm}}}

\end{document}
