\documentclass[a4paper,11pt]{article}
\usepackage[utf8]{inputenc}
\usepackage[T1]{fontenc}
\usepackage[french]{babel}
\usepackage{geometry}
\usepackage{enumitem}
\usepackage{titlesec}
\usepackage{hyperref}

\geometry{margin=2.5cm}

\titleformat{\section}{\Large\bfseries}{\thesection}{1em}{}
\titleformat{\subsection}{\large\bfseries}{\thesubsection}{1em}{}

\hypersetup{
    colorlinks=true,
    linkcolor=blue,
    urlcolor=blue,
    pdftitle={Questions d'entretien - Professions après BUT GIM},
    pdfauthor={BUT Génie Industriel et Maintenance}
}

\begin{document}

\title{\textbf{Liste de Questions} \\ 
\large Enseignants-Chercheurs \\ 
\normalsize Métiers après BUT Génie Industriel et Maintenance}
\author{Rong ZHOU - Université de Lorraine}
\date{\today}

\maketitle

\vspace{0.5cm}

\section{Parcours et Expérience Professionnelle}

\begin{enumerate}[leftmargin=*]
    \item Pouvez-vous nous présenter brièvement votre parcours professionnel et votre domaine de recherche actuel ?
    
\end{enumerate}

\section{Qualités et Compétences Nécessaires}

\begin{enumerate}[leftmargin=*]
    \item Selon votre expérience, quelles sont les principales qualités humaines essentiel pour enseignant-chercheurs ?
    
    \item Et quelles sont les principales qualités techniques essentiel pour enseignant-chercheurs ?
    
    \item Quelles compétences transversales (communication, gestion de projet, travail en équipe, etc.) sont particulièrement valorisées ?
    
    \item Quelles autres compétences sont essentiel pour enseignant-chercheurs ?
\end{enumerate}

\section{Expérience Personnelle et Quotidien Professionnel}

\subsection{Journée type}

\begin{enumerate}[leftmargin=*]
    \item Pouvez-vous décrire une journée type dans votre métier d'enseignant-chercheur ?
    
    \item Quelle est la répartition approximative de votre temps entre l'enseignement, la recherche, l'administration et les autres activités ?
    
    \item Quels types d'interactions avez-vous au quotidien (avec des étudiants, des collègues, des industriels, etc.) ?
\end{enumerate}

\subsection{Ressentis et Réflexions}

\begin{enumerate}[leftmargin=*]
    \item Qu'est-ce que vous appréciez particulièrement dans votre travail ? Ce qui vous motive au quotidien ?
    
    \item Quels sont les aspects les plus difficiles de votre profession ?

    \item Comment décririez-vous votre charge de travail ? Est-elle équilibrée ou y a-t-il des périodes particulièrement intenses ?
    
    \item Quel est l'équilibre entre vie professionnelle et vie personnelle dans votre métier ?
    
    \item Quelles sont les principales sources de satisfaction dans votre travail ?
    
    \item Y-a-t-il des difficultés ou frustrations particulières que vous rencontrez ?
\end{enumerate}

\section{Évolutions de Carrière Possibles}

\begin{enumerate}[leftmargin=*]
    \item Quelles sont les principales évolutions de carrière possibles pour un enseignant-chercheur ?
    
    \item Connaissez-vous des enseignants-chercheurs qui ont évolué vers d'autres métiers (médiateur scientifique, ingénieur R\&D, consultant, etc.) ?
    
    \item Quelles formations complémentaires ou qualifications sont nécessaires pour progresser dans la carrière académique ?
\end{enumerate}

\section{De BUT GIM aux enseignant-chercheurs : Conseils et Perspectives}

\begin{enumerate}[leftmargin=*]
    
    \item Quels conseils donneriez-vous à un étudiant qui vise à devenir enseignant-chercheur?

    \item Y a-t-il des aspects importants que nous n'avons pas abordés et que vous souhaiteriez partager ?
\end{enumerate}

\vspace{1cm}

\section*{}

Merci beaucoup pour votre participation et le temps que vous consacréz à cet entretien. Vos réponses nous seront précieuses pour guider les étudiants dans leurs choix de carrière.

\vspace{0.5cm}

\noindent\rule{\textwidth}{0.4pt}

\vspace{0.3cm}

\noindent\textit{Date de l'entretien : \underline{\hspace{4cm}}}

\noindent\textit{Nom de l'enseignant-chercheur : \underline{\hspace{6cm}}}

\noindent\textit{Institution : \underline{\hspace{8cm}}}

\end{document}
