\documentclass[a4paper,11pt]{article}
\usepackage[utf8]{inputenc}
\usepackage[T1]{fontenc}
\usepackage[french]{babel}
\usepackage{geometry}
\usepackage{enumitem}
\usepackage{titlesec}
\usepackage{hyperref}

\geometry{margin=2.5cm}

\titleformat{\section}{\Large\bfseries}{\thesection}{1em}{}
\titleformat{\subsection}{\large\bfseries}{\thesubsection}{1em}{}

\hypersetup{
    colorlinks=true,
    linkcolor=blue,
    urlcolor=blue,
    pdftitle={Questions d'entretien - Professions après BUT GIM},
    pdfauthor={BUT Génie Industriel et Maintenance}
}

\begin{document}

\title{\textbf{Liste de questions} \\
\large Enseignant·e·s-chercheur·se·s \\
\normalsize Métiers après BUT Génie Industriel et Maintenance}
\author{Rong ZHOU - Université de Lorraine}
\date{\today}

\maketitle

\vspace{0.5cm}

\section{Parcours et expérience professionnelle}

\begin{enumerate}[leftmargin=*]
    \item Pouvez-vous nous présenter brièvement votre parcours professionnel et votre domaine de recherche actuel ?

\end{enumerate}

\section{Qualités et compétences nécessaires}

\begin{enumerate}[leftmargin=*]
    \item Selon votre expérience, quelles sont les principales qualités humaines essentielles pour les enseignant·e·s-chercheur·se·s ?

    \item Quelles compétences techniques jugez-vous indispensables dans votre domaine, et comment conseillez-vous de les acquérir ou de les approfondir ?

    \item Quelles compétences transversales (communication, gestion de projet, travail en équipe, etc.) sont particulièrement valorisées ? Pouvez-vous donner des exemples concrets ?

    \item Y a-t-il d'autres compétences, formelles ou informelles, que vous considérez importantes pour réussir dans ce rôle ?
\end{enumerate}

\section{Expérience personnelle et quotidien professionnel}

\subsection{Journée type}

\begin{enumerate}[leftmargin=*]
    \item Pouvez-vous décrire une journée type dans votre métier d'enseignant-chercheur ?

    \item Quelle est la répartition approximative de votre temps entre l'enseignement, la recherche, l'administration et les autres activités ?

    \item Quels types d'interactions avez-vous au quotidien (avec des étudiants, des collègues, des industriels, etc.) ?
\end{enumerate}

\subsection{Ressentis et réflexions}

\begin{enumerate}[leftmargin=*]
    \item Qu'appréciez-vous le plus dans votre travail et qu'est-ce qui vous motive au quotidien ? Pouvez-vous illustrer par un exemple concret ?

    \item Quels sont, selon vous, les aspects les plus difficiles de votre profession, et comment les gérez-vous ?

    \item Comment décririez-vous votre charge de travail ? Est-elle généralement équilibrée ou soumise à des périodes intenses ? Comment les gérez-vous ?
    
    \item Quel est l'équilibre entre vie professionnelle et vie personnelle dans votre métier et quelles stratégies utilisez-vous pour le préserver ?
    
    \item Quelles sont les principales sources de satisfaction dans votre travail ?
    
    \item Rencontrez-vous des difficultés ou des frustrations particulières ? Si oui, quelles pistes ou ressources recommandez-vous pour les surmonter ?
\end{enumerate}

\section{Évolutions de carrière possibles}

\begin{enumerate}[leftmargin=*]
    \item Quelles sont les principales évolutions de carrière possibles pour un enseignant-chercheur ?

    \item Connaissez-vous des enseignants-chercheurs qui ont évolué vers d'autres métiers (médiateur scientifique, ingénieur R\&D, consultant, etc.) ?

    \item Quelles formations complémentaires ou quelles qualifications sont nécessaires pour progresser dans la carrière académique ?
\end{enumerate}

\section{Du BUT GIM au métier d'enseignant·e·s-chercheur·se·s : conseils et perspectives}

\begin{enumerate}[leftmargin=*]
    \item Quels conseils donneriez-vous à un·e étudiant·e issu·e d'un BUT GIM qui souhaite s'orienter vers la carrière d'enseignant·e·chercheur·se ? (compétences à développer, expériences à privilégier, démarches à entreprendre)

    \item Y a-t-il des éléments importants que nous n'avons pas abordés et que vous souhaiteriez partager pour aider les étudiant·e·s dans leur projet professionnel ?
\end{enumerate}

\vspace{1cm}

\section*{}

Merci beaucoup pour votre participation et le temps que vous consacrez à cet entretien. Vos réponses nous seront précieuses pour guider les étudiants dans leurs choix de carrière.

\vspace{0.5cm}

\noindent\rule{\textwidth}{0.4pt}

\vspace{0.3cm}

\noindent\textit{Date de l'entretien : \underline{\hspace{4cm}}}

\noindent\textit{Nom de l'enseignant·e·chercheur·se : \underline{\hspace{6cm}}}

\noindent\textit{Institution : \underline{\hspace{8cm}}}

\end{document}